\documentclass[11pt,a4paper]{article}
\usepackage[utf8]{inputenc}
\usepackage[T1]{fontenc}
\usepackage{amsmath,amsfonts,amssymb}
\usepackage{graphicx}
\usepackage{subcaption}
\usepackage{booktabs}
\usepackage{geometry}
\usepackage{hyperref}
\usepackage{float}
\usepackage{listings}
\usepackage{xcolor}

\geometry{margin=2.5cm}

% Remove colors from hyperlinks
\hypersetup{
    colorlinks=false,
    pdfborder={0 0 0},
}

% Code listings style
\lstset{
    basicstyle=\ttfamily\small,
    breaklines=true,
    frame=single,
    backgroundcolor=\color{gray!10}
}

\title{Reproducing ``Riding Wavelets: A Method to Discover New Classes of Price Jumps''}
\author{Implementation Report}
\date{\today}

\begin{document}

\maketitle

\begin{abstract}
This report documents the reproduction of the methodology described in ``Riding Wavelets: A Method to Discover New Classes of Price Jumps'' by Aubrun et al. We implement the wavelet-based feature extraction and Kernel PCA for jump classification, analyze the distribution of jump scores, and compare results with and without Scattering Spectra features. Our implementation is tested on multiple datasets including Poland and Hong Kong stock markets.
\end{abstract}

\tableofcontents
\newpage

\section{Introduction}

This report presents our implementation of the jump detection and classification methodology from Aubrun et al. The paper introduces an unsupervised framework leveraging wavelet coefficients to classify financial price jumps based on their reflexivity, mean-reversion, and trend characteristics.

Our implementation includes:
\begin{itemize}
    \item Jump detection using 4-sigma threshold on standardized returns
    \item Wavelet scattering coefficient extraction
    \item Kernel PCA for dimensionality reduction
    \item Classification into endogenous, exogenous, and anticipatory jumps
    \item Analysis of mean-reversion and trend directions
    \item Comparison with and without Scattering Spectra features
\end{itemize}

\section{Methodology}

\subsection{Jump Detection}

Following the paper's methodology, we compute standardized returns:
\begin{equation}
x(t) = \frac{r(t)}{f(t)\sigma(t)}
\end{equation}
where $r(t)$ are 1-minute returns, $f(t)$ is the intraday volatility pattern (U-shape), and $\sigma(t)$ is the local volatility estimated using exponentially weighted moving standard deviation.

Jumps are detected when $|x(t)| > 4$, corresponding to a 4-sigma deviation. We filter trading hours to exclude the first and last 60 minutes of each trading day (10:30-15:00 window).

\subsection{Wavelet Feature Extraction}

We implement the wavelet scattering representation $\Phi(x)$ consisting of:
\begin{itemize}
    \item Primary coefficients: $W_j \bar{x}(0)$ normalized by scale energy
    \item Second-order coefficients: $W_{j_2}|W_{j_1}x|(0)$ for $j_1 < j_2$
\end{itemize}

For $J=3$ scales, this yields 12 features (real and imaginary parts of complex coefficients).

\subsection{Scattering Spectra}

We also implement the Scattering Spectra (SS) features as described in Morel et al.:
\begin{itemize}
    \item $\Phi_1[j]$: Low-moment kurtosis measure
    \item $\Phi_2[j]$: Average volatility at scale $2^j$
    \item $\Phi_3[j,j']$: Multi-scale skewness (complex)
    \item $\Phi_4[j_1,j'_1,j_2]$: Multi-scale kurtosis via second-order scattering (complex)
\end{itemize}

\subsection{Kernel PCA and Classification}

We apply Kernel PCA (RBF kernel) to the wavelet features to obtain a low-dimensional embedding. The first principal component $D_1$ captures volatility asymmetry (reflexivity). We also compute handcrafted features:
\begin{itemize}
    \item $D_2$: Mean-reversion = $x(-1) - x(+1)$
    \item $D_3$: Trend = $x(-1) + x(+1)$
\end{itemize}

\section{Results}

\subsection{Jump Score Distribution}

Figure~\ref{fig:jump_distribution} shows the distribution of standardized returns $x(t)$ and the fit to a Gumbel distribution for the upper tail (quantiles $> 0.28$). The Q-Q plot in Figure~\ref{fig:qq_plot} confirms the Gumbel distribution is a good fit for the tail region.

\begin{figure}[H]
\centering
\begin{subfigure}{0.48\textwidth}
    \centering
    \includegraphics[width=\textwidth]{figures/jump_density_poland_distribution_signed.pdf}
    \caption{Distribution of signed standardized returns}
    \label{fig:dist_signed}
\end{subfigure}
\begin{subfigure}{0.48\textwidth}
    \centering
    \includegraphics[width=\textwidth]{figures/jump_density_poland_distribution_abs.pdf}
    \caption{Distribution of absolute standardized returns with Gumbel fit}
    \label{fig:dist_abs}
\end{subfigure}
\caption{Distribution of standardized returns $x(t)$ for Poland dataset}
\label{fig:jump_distribution}
\end{figure}

\begin{figure}[H]
\centering
\includegraphics[width=0.7\textwidth]{figures/jump_density_poland_qq_plot.pdf}
\caption{Q-Q plot comparing empirical quantiles to Gumbel distribution (quantiles 0.28-1.0)}
\label{fig:qq_plot}
\end{figure}

\subsection{PCA Directions}

\subsubsection{D1: Reflexivity}

Figure~\ref{fig:d1_asymmetry} shows the correlation between $D_1$ (reflexivity) and the asymmetry measure $\mathcal{A}_{\text{jump}} = (\mathcal{A}_{>} - \mathcal{A}_{<})/(\mathcal{A}_{>} + \mathcal{A}_{<})$. The strong positive correlation confirms that $D_1$ captures volatility asymmetry.

\begin{figure}[H]
\centering
\includegraphics[width=0.7\textwidth]{figures/reproduce_poland_D1_asymmetry.pdf}
\caption{D1 (Reflexivity) vs Asymmetry measure. Positive correlation confirms D1 captures time-asymmetry of volatility.}
\label{fig:d1_asymmetry}
\end{figure}

\subsubsection{D1 vs D2: Mean-Reversion}

Figure~\ref{fig:d1_d2} shows the 2D projection of jumps onto the reflexivity ($D_1$) and mean-reversion ($D_2$) directions. This reveals different classes of jumps: endogenous (negative $D_1$), exogenous (positive $D_1$), and mean-reverting (large $|D_2|$).

\begin{figure}[H]
\centering
\includegraphics[width=0.7\textwidth]{figures/reproduce_poland_fig5_mr.pdf}
\caption{2D projection: Reflexivity ($D_1$) vs Mean-Reversion ($D_2$)}
\label{fig:d1_d2}
\end{figure}

\subsubsection{D1 vs D3: Trend}

Figure~\ref{fig:d1_d3} shows the projection onto reflexivity and trend directions, revealing trend-aligned and trend-anti-aligned jumps.

\begin{figure}[H]
\centering
\includegraphics[width=0.7\textwidth]{figures/reproduce_poland_fig6_tr.pdf}
\caption{2D projection: Reflexivity ($D_1$) vs Trend ($D_3$)}
\label{fig:d1_d3}
\end{figure}

\subsection{Average Profiles}

Figure~\ref{fig:profiles_d1} shows average temporal profiles along the $D_1$ direction, confirming that:
\begin{itemize}
    \item Negative $D_1$ (left): Anticipatory jumps with pre-jump activity
    \item Near zero $D_1$ (center): Endogenous jumps with symmetric profiles
    \item Positive $D_1$ (right): Exogenous jumps with post-jump activity
\end{itemize}

\begin{figure}[H]
\centering
\includegraphics[width=0.9\textwidth]{figures/reproduce_poland_profile_D1_reflexivity.pdf}
\caption{Average profiles along D1 (Reflexivity) direction, sliced into quantile bins}
\label{fig:profiles_d1}
\end{figure}

Similar profiles for $D_2$ and $D_3$ are shown in Figures~\ref{fig:profiles_d2} and~\ref{fig:profiles_d3}.

\begin{figure}[H]
\centering
\begin{subfigure}{0.48\textwidth}
    \centering
    \includegraphics[width=\textwidth]{figures/reproduce_poland_profile_D2_mean_reversion.pdf}
    \caption{Mean-Reversion ($D_2$)}
    \label{fig:profiles_d2}
\end{subfigure}
\begin{subfigure}{0.48\textwidth}
    \centering
    \includegraphics[width=\textwidth]{figures/reproduce_poland_profile_D3_trend.pdf}
    \caption{Trend ($D_3$)}
    \label{fig:profiles_d3}
\end{subfigure}
\caption{Average temporal profiles along D2 and D3 directions}
\label{fig:profiles_d2_d3}
\end{figure}

\subsection{Example Jumps}

Figure~\ref{fig:example_jumps} shows examples of endogenous and exogenous jumps from the dataset, including their normalized profiles and actual price/return series.

\begin{figure}[H]
\centering
\includegraphics[width=0.9\textwidth]{figures/jump_density_poland_examples.pdf}
\caption{Example endogenous and exogenous jumps from the dataset. Top row shows $|x(t)|$ profiles, bottom row shows $x(t)$ profiles.}
\label{fig:example_jumps}
\end{figure}

\subsection{Scattering Spectra Comparison}

We compare results with and without Scattering Spectra features. Figure~\ref{fig:ss_comparison_d1} shows that including SS features does not significantly alter the $D_1$ direction (correlation $> 0.95$), suggesting the base wavelet features are sufficient for capturing reflexivity.

\begin{figure}[H]
\centering
\begin{subfigure}{0.48\textwidth}
    \centering
    \includegraphics[width=\textwidth]{figures/poland_comparison_D1_scatter.pdf}
    \caption{D1 scatter plot comparison}
    \label{fig:ss_comparison_d1}
\end{subfigure}
\begin{subfigure}{0.48\textwidth}
    \centering
    \includegraphics[width=\textwidth]{figures/poland_comparison_D1_distribution.pdf}
    \caption{D1 distributions}
    \label{fig:ss_comparison_d1_dist}
\end{subfigure}
\caption{Comparison of D1 with and without Scattering Spectra}
\label{fig:ss_comparison_d1_full}
\end{figure}

Similar comparisons for $D_2$ and $D_3$ are shown in Figures~\ref{fig:ss_comparison_d2} and~\ref{fig:ss_comparison_d3}.

\begin{figure}[H]
\centering
\begin{subfigure}{0.48\textwidth}
    \centering
    \includegraphics[width=\textwidth]{figures/poland_comparison_D2_scatter.pdf}
    \caption{D2 scatter plot comparison}
    \label{fig:ss_comparison_d2}
\end{subfigure}
\begin{subfigure}{0.48\textwidth}
    \centering
    \includegraphics[width=\textwidth]{figures/poland_comparison_D3_scatter.pdf}
    \caption{D3 scatter plot comparison}
    \label{fig:ss_comparison_d3}
\end{subfigure}
\caption{Comparison of D2 and D3 with and without Scattering Spectra}
\label{fig:ss_comparison_d2_d3}
\end{figure}

\section{Multi-Stock Analysis}

We analyze the dependence on the number of stocks and data quantity. Figure~\ref{fig:multi_stock} shows results for different subsets of stocks.

\begin{figure}[H]
\centering
\begin{subfigure}{0.48\textwidth}
    \centering
    \includegraphics[width=\textwidth]{figures/reproduce_poland_5_Stocks_Poland_fig5_mr.pdf}
    \caption{5 stocks}
\end{subfigure}
\begin{subfigure}{0.48\textwidth}
    \centering
    \includegraphics[width=\textwidth]{figures/reproduce_poland_30_Stocks_Poland_fig5_mr.pdf}
    \caption{30 stocks}
\end{subfigure}
\caption{D1 vs D2 projections for different numbers of stocks (Poland dataset). The patterns are consistent across different data quantities.}
\label{fig:multi_stock}
\end{figure}

\section{Discussion}

Our implementation successfully reproduces the main findings of the paper:
\begin{enumerate}
    \item The jump score $|x(t)|$ follows a Gumbel distribution in the tail region
    \item The first PCA direction $D_1$ captures volatility asymmetry (reflexivity)
    \item Mean-reversion ($D_2$) and trend ($D_3$) are important additional features
    \item The methodology identifies endogenous, exogenous, and anticipatory jumps
\end{enumerate}

The inclusion of Scattering Spectra features does not significantly alter the discovered directions, suggesting that the base wavelet scattering coefficients are sufficient for this classification task.

\section{Conclusion}

We have successfully implemented and validated the wavelet-based jump classification methodology. The code is available in the \texttt{notebooks/jump/} directory and can be applied to different markets (Poland, Hong Kong) and datasets.

Future work could include:
\begin{itemize}
    \item Extension to co-jump analysis
    \item Application to higher frequency data
    \item Comparison with other classification methods
    \item Analysis of market-specific patterns
\end{itemize}

\end{document}

