\documentclass[12pt,a4paper]{article}
\usepackage[utf8]{inputenc}
\usepackage[T1]{fontenc}
\usepackage[english]{babel}
\usepackage{amsmath, amsfonts, amssymb, amsthm}
\usepackage{geometry}
\usepackage{graphicx}
\usepackage{fancyhdr}
\usepackage{hyperref}
\usepackage{tikz}
\usepackage{xcolor}
\usepackage{enumitem}
\usepackage{tcolorbox}
\usepackage{physics}
\usepackage{float}
\usepackage{booktabs}
\newcommand{\RR}{\mathbb{R}}
\newtheoremstyle{exerciseStyle}% name
  {3pt}% Space above
  {3pt}% Space below
  {\normalfont}% Body font
  {}% Indent amount
  {\bfseries}% Theorem head font
  {}% Punctuation after theorem head
  {\newline}% Space after theorem head
  {}% Theorem head spec (can be left empty)

\theoremstyle{exerciseStyle}
\newtheorem{exercise}{Question}
\newtheorem{solution}{Answer}
\geometry{
    left=2.5cm,
    right=2.5cm,
    top=2.5cm,
    bottom=2.5cm,
}

\hypersetup{
    colorlinks=true,
    linkcolor=cyan!60!blue,
    urlcolor=magenta,
    citecolor=orange!70!red
}

% Header and Footer Settings
\pagestyle{fancy}
\fancyhf{}
\lhead{\textcolor{teal!80!black}{Math Report}}
\rhead{\textcolor{gray}{\thepage}}
\lfoot{MVA Time Series $\cdot$ \texttt{timeseries}}
\rfoot{\today}

% Title Page
\begin{document}
\begin{titlepage}
    \begin{center}
        \vspace*{2cm}
        \Huge\bfseries
        \textcolor{teal!70!black}{Mathematics Class Report} \\[0.5cm]
        \Large
        % Insert Assignment Name
        Assignment Title \\
        \vspace{1cm}
        \includegraphics[width=0.25\textwidth]{example-image} % Replace with your logo or image file if desired
        \vspace{1.5cm}
        
        \begin{large}
        Your Name \\
        Course: MVA -- Time Series Analysis\\
        \vspace{0.5cm}
        Instructor: Dr.~\_\_\_\_\_\\
        Deadline: \underline{\hspace{2cm}}\\
        \end{large}
        \vfill
        \textsc{Academic Year: 2023--2024}
    \end{center}
\end{titlepage}

% Table of Contents
\tableofcontents
\thispagestyle{empty}
\newpage

% Useful boxes
\newtcolorbox{theobox}[1][]{colback=yellow!10!white, colframe=orange!80!black, fonttitle=\bfseries, title=Theorem~#1}
\newtcolorbox{defbox}[1][]{colback=blue!5!white, colframe=blue!80!black, fonttitle=\bfseries, title=Definition~#1}
\newtcolorbox{exbox}[1][]{colback=green!5!white, colframe=green!80!black, fonttitle=\bfseries, title=Example~#1}

% Main Section Templates

\section{Introduction}
\begin{itemize}[leftmargin=1.5cm]
    \item Briefly describe the objective of your report.
    \item Provide context and motivation.
\end{itemize}

\section{Theory}
\begin{defbox}
Here, you can state and define important mathematical concepts.
\end{defbox}
\begin{theobox}
Place for theorems, lemmas, properties, etc.
\end{theobox}

\begin{exercise}
    
Consider the following Lasso regression:
\begin{equation}\label{eq:lasso}
    \min_{\beta\in\RR^p} \frac{1}{2}\norm{y-X\beta}^2_2 \quad + \quad \lambda \norm{\beta}_1
\end{equation}
where $y\in\RR^n$ is the response vector, $X\in\RR^{n\times p}$ the design matrix, $\beta\in\RR^p$ the vector of regressors and $\lambda>0$ the smoothing parameter.

Show that there exists $\lambda_{\max}$ such that the minimizer of~\eqref{eq:lasso} is $\mathbf{0}_p$ (a $p$-dimensional vector of zeros) for any $\lambda > \lambda_{\max}$. 
\end{exercise}

\begin{solution}  % ANSWER HERE

\begin{equation}
    \lambda_{\max} = \max_{j\in\{1,\ldots,p\}} \left|\sum_{i=1}^n X_{ij}y_i \right|
\end{equation}

Proof: Let \(f(\beta) = \frac{1}{2}\norm{y-X\beta}^2_2 \). The function is twice differentiable everywhere, and the Hessian is \(\nabla^2 f(\beta) = X^TX \succeq 0\).
Therefore \(f\) is convex. We know the function satisfies, for every pair of points \((x,y)\):
\(f(y) \geq f(x) + \langle\grad f(x), y-x\rangle\), and so by letting \(y=\beta\), \(x=0\), and computing the gradient of the function at 0 which is \(\grad f(0) = -X^Ty\)
we finally get the identity:

\begin{equation}
    f(\beta) \geq f(0) - \langle X^Ty , \beta \rangle
\end{equation}

Now, if we let \(\lambda > \lambda_{\max}\), we have that \(\lambda|\beta_i|\) is strictly greater than the absolute value of the \(i\)'th component of the dot product (this is true by definition of \(\lambda_{\max}\)), and therefore by adding all the terms we get
\[f(\beta) + \lambda\norm{\beta}_1 > f(0) \quad \forall \beta \in \mathbb{R}^p \]

This proves what we wanted.

\end{solution}


\section{Methodology}
\begin{itemize}
    \item Explain the approaches, algorithms, or tools used.
    \item Illustrate with formulas:
    \[
        x(t) = \sum_{i=1}^n a_i \cdot \phi_i(t)
    \]
\end{itemize}

\section{Results}
\begin{figure}[H]
    \centering
    \includegraphics[width=0.6\textwidth]{example-image}
    \caption{Sample Plot -- describe your results here.}
\end{figure}

% For tables
\begin{table}[H]
\centering
\begin{tabular}{lll}
\toprule
Parameter & Value & Description \\
\midrule
$a$ & 2 & Slope coefficient \\
$b$ & 3 & Intercept \\
\bottomrule
\end{tabular}
\caption{An example table}
\end{table}

\section{Discussion}
\begin{itemize}
    \item Interpret and critique your results.
    \item Discuss any surprising findings.
\end{itemize}

\section{Conclusion}
\begin{itemize}
    \item Summarize what you learned.
    \item Propose future directions or applications.
\end{itemize}

\section*{References}
\begin{enumerate}
    \item Author, ``Title of Book or Article,'' Journal/Publisher, Year.
    \item ...
\end{enumerate}

\end{document}
